\chapter{Control Architecture}
\section{Introduction}
Vukobratovic \cite{Vuk1970} was one of the first researchers involved in the stability of bipedal robots, followed by \cite{Kaj2001} and \cite{Kim2004}. In all their studies, the biped robot was usually represented by a planar inverted pendulum with the base representing the foot and the ankle joint. And in the latest control strategies, researchers divide robot balance control into the hip strategy and the ankle strategy \textcolor{red}{REFERENCIA Y EXTENDER MÁS}. The basis of both stategies are close to the ZMP areas explained in previous sections. When the robot is in a stable posture and a disturbance is applied, depending the magnitude or the application point of that distrubance, the robot will react different. If the change of the ZMP position remains in a stable area, the control will react by the motion of the ankle joints to recover the robot balance. Nevertheless, if the ZMP position reaches an uncertain-stability area, it will be also necessary to move the hip joints to recover balance. Even a gait will be necessary if the loss of stability is unavoidable. 

A humanoid is an electromechanical system, so it should have all type of errors: structure flexion, small blacklash between motion parts, etc. Also it will operate in a co-existing environment with humans, so the disturbances are unexpected at any time. Therefore, the Stabilizer is an essential element to provide stable human-like walking of a humanoid robot. The Stabilizer should perform two basic operations:
1. When the humanoid robot walks, it should correct the robot’s walking trajectory in order to provide the secure position at any time of its motion.
2. When the humanoid robot has stopped, it should control its posture.

Thus, the Stabilizer can be decoupled into ZMP and Attitude controllers. This Master Thesis will deal with the issue of maintaining an upright posture while the robot is in a static position, without following a motion pattern.


\input{05_control_architecture/00_inv_pendulum_model}
\section{Feedback in state space. The Linear Quadratic Regulator}

The quadratic optimal control method is one of the control methods applied in state space systems and it provides a systematic way of computing the state feedback control gain matrix \cite{Ogata}.
Given the state space system equation
\begin{equation}
\dot{x} = Ax+Bu ,
\label{eq:sseq}
\end{equation}
the LQR determines the matrix $K$ of the optima control vector
\begin{equation}
u(t) = -Kx(t)
\label{eq:control}
\end{equation}
so as to minimize the performance index
\begin{equation}
J = \int_{0}^{\infty}(x^{T}Qx+u^{T}Ru) dt
\end{equation}

where $Q$ is a positive-definite (or positive-semidefinite) Hermitian or real symmetric matrix and $R$ is a positive-definite Hermitian or real symmetric matrix. Note that the matrices $Q$ and $R$ determine the relative importance of the error and the expenditure of the energy of the control signals.
The linear control law given by equation \ref{eq:control} is the optimal control law. Therefore, if the unknown elements of the matrix $K = [K_1 \quad K_2]$ are determined so as to minimize the performance index, then $u(t) = -Kx$  is optimal for any initial state $x(0)$. The block diagram showing the optimal configuration for the single inverted pendulum system is presented in Figure \ref{fig:block_diagram}. The controller maintain desired ($x_{ZMP}$) position of the single pendulum close to zero. Thus, the reference input of the control system in Figure [REF] is not zero.?????
\begin{figure}[!hbt]
\centering
\includegraphics[scale=0.4]{diagrama.pdf}
\caption{Block diagram}
\label{fig:block_diagram}
\end{figure}

The state space representation \ref{eq:state_space}, \ref{eq:state_space_out} is a controllable canonical form that is important for the LQR controller design. It is desired to keep the actual ZMP, measured and computed by force-torque sensors located in the feet of the humanoid robot, close to its stable reference position as was discussed in previous sections. As the system is a type 0 plant, it is necessary to insert an integrator in order to design a ZMP servo control system (type 1) and remove the steady state error. Therefore, we feed the output signal $y$ (which indicates the real ZMP) back to the input and an integrator in the feedforward path as is shown in Figure \ref{fig:diagrama_int}.

\begin{figure}[!hbt]
\centering
\includegraphics[scale=0.4]{diagrama_int.pdf}
\caption{ZMP LQR control system.}
\label{fig:diagrama_int}
\end{figure}

Thus, referring equations \ref{eq:state_space} and \ref{eq:state_space_out} and Figure \ref{fig:diagrama_int} and considering the actual ZMP position as the output of the system and $r$ as the reference input signal we obtain the equations for the closed loop system as follows:
\begin{equation}
\dot{x} = \textbf{A}\textbf{x} + \textbf{B}u
\end{equation}

\begin{equation}
y = \textbf{C}\textbf{x} + \textbf{D}u
\end{equation}

\begin{equation}
u = - \textbf{K}\textbf{x} + K_i z
\end{equation}

\begin{equation}
\dot{z} = r - y = r - (\textbf{C}\textbf{x} + \textbf{D}u)
\end{equation}

For the type 1 servo system, the state error equation is given by:
\begin{equation}
\begin{bmatrix}
\dot{\textbf{x}}\\
\dot{z}
\end{bmatrix} = 
\begin{bmatrix}
\textbf{A} & \textbf{0}\\
\textbf{-C} & 0
\end{bmatrix}
\begin{bmatrix}
\textbf{x}\\
z
\end{bmatrix} + 
\begin{bmatrix}
\textbf{B}\\
0
\end{bmatrix}
u
\end{equation}
and the control signal $u$ is given by:
\begin{equation}
u = \begin{bmatrix}
-\textbf{K} & K_i
\end{bmatrix}
\begin{bmatrix}
\textbf{x}\\
z
\end{bmatrix}
\end{equation}

The optimum $K$ matrix is obtained from equations \ref{eq:Kcont} and \ref{eq:Kdisc}.
\begin{equation}
K = R^{-1}B^{T}P \quad (continuous \quad case)
\label{eq:Kcont}
\end{equation}
\begin{equation}
K = (R + B^{T}PB)^{-1}B^{T}PA \quad (discrete \quad case)
\label{eq:Kdisc}
\end{equation}

where $P$ is a positive-definite Hermitian or real symmetric matrix and it is necessary to compute the algebraic Ricatti Equation
\begin{equation}
P \rightarrow A^{T}P+PA-PBR^{-1}B^{T}P+Q = 0 \quad (continuous \quad case)
\end{equation}
\begin{equation}
P \rightarrow A^{T}PA+P-A^{T}PB(R+B^{T}PB)^{-1}B^{T}PA+Q = 0 \quad (discrete \quad case)
\end{equation}


In order to obtain the controller design for further simulations and experiments, the following mechanical parameters of the inverted pendulum (corresponding to Rh-2 humanoid robot) were taken: $m$ = 62.416 kg, $l$=1.03 m, $k$=0.1. \textcolor{red}{Valor y unidades de stiffness???}. For the optimum response of the control system, it is suggested to take $Q = C^{T}C = \begin{bmatrix}
5.037 \cdot 10^{-8} & 0\\
0 & 0
\end{bmatrix}$ and $R = 1 $. 
%%% AÑADIR SOLO SI SE PRUEBA QUE ES MAS RAPIDO %%%%
%But for a faster response of the control, we take $Q = C^{T}C = \begin{bmatrix}
%1000 & 0\\
%0 & 0
%\end{bmatrix}$.  
After the LQR controller was designed, the control gains matrix $K = [18.89 \quad 6.14]$ was obtained using a sample time $T = 0.001$ s.












\input{05_control_architecture/02_stabilizer}


\section{Control strategy}
As previously mentioned, the human body can be divided in three basic planes: sagital, transversal and frontal. Frontal and sagital planes are useful to simplify the controller and it can be decoupled into two 2D linear pendulums instead of one 3D pendulum.

In the frontal control, the pendulum will rotate around the X axis. Sagital joint ankles and sagital hip will move.
In the sagital control, the pendulum will rotate around the Y axis. Frontal joint ankles will move.
 


The phenomenon of locomotion can be understood more easily if we study the motion of a human in three basic planes: sagittal, transversal and frontal (Fig. 5.1). It is important to mention that the most important motions occur in the sagittal plane because it coincides with the main walking direction. Motions in transversal and frontal planes are not so radical. The joints responsible for motion in these two plains (for the most part in the frontal plane) help significantly in the stabilization of the locomotion cycle.
